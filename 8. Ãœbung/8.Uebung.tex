\documentclass[12pt]{article}
\usepackage{amsfonts}
\usepackage{fancyhdr}
\usepackage[a4paper, top=2.5cm, bottom=2.5cm, left=2.2cm, right=2.2cm]{geometry}
\usepackage{times}
\usepackage{amsmath}
\usepackage{changepage}
\usepackage{amssymb}
\usepackage{graphicx}%
\setcounter{MaxMatrixCols}{30}
\newtheorem{theorem}{Theorem}
\newtheorem{acknowledgement}[theorem]{Acknowledgement}
\newtheorem{algorithm}[theorem]{Algorithm}
\newtheorem{axiom}{Axiom}
\newtheorem{case}[theorem]{Case}
\newtheorem{claim}[theorem]{Claim}
\newtheorem{conclusion}[theorem]{Conclusion}
\newtheorem{condition}[theorem]{Condition}
\newtheorem{conjecture}[theorem]{Conjecture}
\newtheorem{corollary}[theorem]{Corollary}
\newtheorem{criterion}[theorem]{Criterion}
\newtheorem{definition}[theorem]{Definition}
\newtheorem{example}[theorem]{Example}
\newtheorem{exercise}[theorem]{Exercise}
\newtheorem{lemma}[theorem]{Lemma}
\newtheorem{notation}[theorem]{Notation}
\newtheorem{problem}[theorem]{Problem}
\newtheorem{proposition}[theorem]{Proposition}
\newtheorem{remark}[theorem]{Remark}
\newtheorem{solution}[theorem]{Solution}
\newtheorem{summary}[theorem]{Summary}
\usepackage{enumitem}
\usepackage[utf8]{inputenc}
\newenvironment{proof}[1][Proof]{\textbf{#1.} }{\ \rule{0.5em}{0.5em}}
\usepackage{tikz}
\usetikzlibrary{positioning,chains,fit,shapes,calc,arrows,patterns,external,shapes.callouts,graphs}
\usepackage{graphicx}
\usepackage{wrapfig}
\usepackage{float}
\usepackage{datetime}
\usepackage{ifthen}
\newdateformat{specialdate}{\twodigit{\THEDAY}.\twodigit{\THEMONTH}.\THEYEAR}
\usepackage[ngerman]{babel}
\usepackage{rotating}
\newcommand{\positiv}{~~\llap{\color{green!50!black}\textbf{+}\color{black}}~~}
\newcommand{\negativ}{~~\llap{\color{red}\textbf{-}\color{black}}~~}


\newcommand{\Q}{\mathbb{Q}}
\newcommand{\R}{\mathbb{R}}
\newcommand{\C}{\mathbb{C}}
\newcommand{\Z}{\mathbb{Z}}

\usepackage{pifont}% http://ctan.org/pkg/pifont
\newcommand{\cmark}{\ding{51}}%
\newcommand{\xmark}{\ding{55}}%

\begin{document}
	
	\title{8. Übung}
	\author{Timo Bergerbusch 344408}
	\date{\specialdate\today}
	\maketitle
	
	\section*{Aufgabe 1}
	\subsection*{a)}
	\begin{itemize}
		\item Reihenfolge der ersten $k$ Indices
		\item Zielfunktionswert
	\end{itemize}
	
	\subsection*{b)}
	\begin{tabular}{l | l}
		Vorteile attributiv & Vorteile explizit \\ \hline
		\positiv wenig Speicheraufwand	& \positiv nur besuchte Lösungen\\
		\negativ teil ähnliche Lösungen werden als bereits betrachtet Eingestuft & \negativ Speicheraufwendig \\
		\negativ geeignete Kriterien sind nicht offensichtlich & \\
	\end{tabular}

	\subsection*{c)}
	\begin{itemize}
		\item $EXCHANGE(v,w)$ nach eben jenem
		\item $RELOCATE(v,v_{+})$ nach $RELOCATE(v,w)$
		\item $2\text{-Opt}(v,v_{+}) \text{ und } 2\text{-Opt}(w_{-},w)$ nach $2\text{-Opt}(v,w)$
	\end{itemize}

	\subsection*{d)}
	\begin{itemize}
		\item Zulässigkeit
		\item planar
		\item "relativ" gradlinig (Lager-Beispiel)
	\end{itemize}

	\subsection*{e)}
	\begin{itemize}
		\item planar $\Rightarrow$ iterieren nach einem gewissen Muster
		\item 
		\item Iterative-Insertion
	\end{itemize}

	\section*{Aufgabe 2}
	
	\subsubsection*{Initial:}
	Gedächtnis: $\{(x,y,z)\mid x \text{ index, } y \text{ Attribut, } z \text{ \# iterationen}\}:=\emptyset$
	
	\subsubsection*{Iteration 1 (x=1):}
	\begin{figure}[H]
		
		Zielfunktionswert $c(1)=50$\\
		Schritte ohne Verbesserung: 0\\
		\begin{tabular}{r | cc}
			Nachbar $x^\prime$ & 2 & 5 \\ \hline
			Kosten $c(x^\prime)$ & 51 & 52 \\ \hline
			Tabu-Konf. Menge & $\emptyset$ & $\emptyset$ \\
		\end{tabular}\\
		$\Rightarrow $ wähle $x=2$ als nächsten Schritt\\
		\textbf{Gedächtnis}: $\{(1, dec(b_2),0), (2, dec(b_1),0)\}$
	\end{figure}
	
	\subsubsection*{Iteration 2 (x=2):}
	\begin{figure}[H]
		\centering
		bester Zielfunktionswert $c(1)=51$\\
		Schritte ohne Verbesserung: 1\\
		\begin{tabular}{r | cccc}
			Nachbar $x^\prime$   & 1 		& 3		& 4 	& 5	\\ \hline
			Kosten $c(x^\prime)$ & 50 		& 46	& 47	& 52 \\\hline
			Tabu-Konf. Menge     & $\{1,2\}$& $\{1\}$& $\emptyset$ & $\emptyset$\\
		\end{tabular}\\
		$\Rightarrow $ wähle $x=4$ als nächsten Schritt\\
		\textbf{Gedächtnis}: $\{(1,dec(b_2),1), (2,dec(b_1),1), (3, dec(b_4),0)\}$
	\end{figure}

	\subsubsection*{Iteration 3 (x=4):}
	\begin{figure}[H]
		\centering
		bester gefundener Zielfunktionswert $c(4)=47$\\
		Schritte ohne Verbesserung: 0\\
		\begin{tabular}{r | cccc}
			Nachbar $x^\prime$   & 2 & 3 & 5 & 6\\ \hline
			Kosten $c(x^\prime)$ & 51 & 46 & 52 & 48 \\\hline
			Tabu-Konf. Menge     & $\{3\}$ & $\{1,3\}$ & $\emptyset$ & $\{3\}$\\
		\end{tabular}\\
		$\Rightarrow $ wähle $x=5$ als nächsten Schritt\\
		\textbf{Gedächtnis}: $\{(3, dec(b_4),1), (4, dec(b_3),0), (5, inc(b_0),0)\}$
	\end{figure}

	\subsubsection*{Iteration 4 (x=5):}
	\begin{figure}[H]
		\centering
		bester gefundener Zielfunktionswert $c(4)=47$\\
		Schritte ohne Verbesserung: 1\\
		\begin{tabular}{r | cccc}
			Nachbar $x^\prime$   & 1 & 2 & 4 & 8\\\hline
			Kosten $c(x^\prime)$ & 50 & 51 & 47 & 61\\\hline
			Tabu-Konf. Menge     & $\{3,4,5\}$ & $\{3,4,5\}$ & $\{4,5\}$ & $\emptyset$ \\
		\end{tabular}\\
		$\Rightarrow $ wähle $x=8$ als nächsten Schritt\\
		\textbf{Gedächtnis}: $\{(3, dec(b_4),2), (4, dec(b_3),1), (5, inc(b_0),1), (6, inc(b_2),0)\}$
	\end{figure}

	\subsubsection*{Iteration 5 (x=8):}
	\begin{figure}[H]
		\centering
		bester gefundener Zielfunktionswert $c(4)=47$\\
		Schritte ohne Verbesserung: 2\\
		\begin{tabular}{r | ccc}
			Nachbar $x^\prime$   & 5 & 6 & 7\\\hline
			Kosten $c(x^\prime)$ & 52 & 48 & 49\\\hline
			Tabu-Konf. Menge     & $\{6\}$ & $\{4,6\}$ & $\emptyset$ \\
		\end{tabular}\\
		$\Rightarrow $ wähle $x=7$ als nächsten Schritt\\
		\textbf{Gedächtnis}: $\{(4, dec(b_3),2), (5, inc(b_0),2), (6, inc(b_2),1), (7,inc(b_4),0)\}$
	\end{figure}
	
	\subsubsection*{Iteration 6 (x=7):}
	\begin{figure}[H]
		\centering
		bester gefundener Zielfunktionswert $c(4)=47$\\
		Schritte ohne Verbesserung: 3\\
		\begin{tabular}{r | ccc}
			Nachbar $x^\prime$   & 3 & 6 & 8\\\hline
			Kosten $c(x^\prime)$ & 46 & 48 & 61\\\hline
			Tabu-Konf. Menge     & $\emptyset$ & $\{6\}$ & $\{7\}$\\
		\end{tabular}\\
		$\Rightarrow $ wähle $x=3$ als nächsten Schritt\\
		\textbf{Gedächtnis}: $\{(6, inc(b_2),2), (7,inc(b_4),1), (8, inc(b_3),0), (9, dec(b_0),0)\}$
	\end{figure}

	\subsubsection*{Iteration 7 (x=3):}
	\begin{figure}[H]
		\centering
		bester gefundener Zielfunktionswert $c(3)=46$\\
		Schritte ohne Verbesserung: 0\\
		\begin{tabular}{r | cccc}
			Nachbar $x^\prime$   & 2 & 4 & 6 & 7\\\hline
			Kosten $c(x^\prime)$ & 51 & 47 & 48 & 49\\\hline
			Tabu-Konf. Menge     & $\emptyset$ & $\{7\}$ & $\{9\}$ & $\{8,9\}$\\
		\end{tabular}\\
		$\Rightarrow $ wähle $x=2$ als nächsten Schritt\\
		\textbf{Gedächtnis}: $\{(7,inc(b_4),2), (8, inc(b_3),1), (9, dec(b_0),1), (10, dec(b_2),0)\}$
	\end{figure}
	 Hier haben wir den optimalen Zielfunktionswert erreicht. Alle 5 weiteren Iterationen sind daher keine Verbesserungen.
	
	
\end{document}



















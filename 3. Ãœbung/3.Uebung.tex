\documentclass[12pt]{article}
\usepackage{amsfonts}
\usepackage{fancyhdr}
\usepackage[a4paper, top=2.5cm, bottom=2.5cm, left=2.2cm, right=2.2cm]{geometry}
\usepackage{times}
\usepackage{amsmath}
\usepackage{changepage}
\usepackage{amssymb}
\usepackage{graphicx}%
\setcounter{MaxMatrixCols}{30}
\newtheorem{theorem}{Theorem}
\newtheorem{acknowledgement}[theorem]{Acknowledgement}
\newtheorem{algorithm}[theorem]{Algorithm}
\newtheorem{axiom}{Axiom}
\newtheorem{case}[theorem]{Case}
\newtheorem{claim}[theorem]{Claim}
\newtheorem{conclusion}[theorem]{Conclusion}
\newtheorem{condition}[theorem]{Condition}
\newtheorem{conjecture}[theorem]{Conjecture}
\newtheorem{corollary}[theorem]{Corollary}
\newtheorem{criterion}[theorem]{Criterion}
\newtheorem{definition}[theorem]{Definition}
\newtheorem{example}[theorem]{Example}
\newtheorem{exercise}[theorem]{Exercise}
\newtheorem{lemma}[theorem]{Lemma}
\newtheorem{notation}[theorem]{Notation}
\newtheorem{problem}[theorem]{Problem}
\newtheorem{proposition}[theorem]{Proposition}
\newtheorem{remark}[theorem]{Remark}
\newtheorem{solution}[theorem]{Solution}
\newtheorem{summary}[theorem]{Summary}
\usepackage{enumitem}
\usepackage[utf8]{inputenc}
\newenvironment{proof}[1][Proof]{\textbf{#1.} }{\ \rule{0.5em}{0.5em}}
\usepackage{tikz}
\usetikzlibrary{positioning,chains,fit,shapes,calc,arrows,patterns,external,shapes.callouts,graphs}
\usepackage{graphicx}
\usepackage{wrapfig}
\usepackage{float}
\usepackage{datetime}
\usepackage{ifthen}
\newdateformat{specialdate}{\twodigit{\THEDAY}.\twodigit{\THEMONTH}.\THEYEAR}
\usepackage[ngerman]{babel}
\usepackage{rotating}


\newcommand{\Q}{\mathbb{Q}}
\newcommand{\R}{\mathbb{R}}
\newcommand{\C}{\mathbb{C}}
\newcommand{\Z}{\mathbb{Z}}

\begin{document}
	
	\title{3. Übung}
	\author{Timo Bergerbusch 344408 \& Marc Burian 344300}
	\date{\specialdate\today}
	\maketitle
	
	
	\section{Aufgabe}
	\subsection{a)}
	
	\begin{figure}[H]
		\centering
		\begin{tabular}{c | c c c c c c c}
			$i$ & 1 & 2 & 3 & 4 & 5 & 6 & 7 \\ \hline
			$p_i$ & 3 & 4 & 7 & 6 & 2 & 1 & 8\\
			$w_i$ & 4 & 2 & 4 & 4 & 3 & 1 & 5\\
			$\frac{p_i}{w_i}$ & 0.75 & 2 & 1.75 & 1.5 & $\frac{2}{3}$ & 1 & 1.6 \\
		\end{tabular}
		 $\overrightarrow{$sort wrt $\frac{p_i}{w_i}}$ 
		 \begin{tabular}{c | c c c c c c c}
		 	$i$ 				&	2	& 3 	& 4 	& 7	& 6 &	1		& 5 			\\ \hline
		 	$p_i$ 				&	4	& 7 	& 6 	& 8 & 1 &	3 		& 2 			\\
		 	$w_i$ 				&	2	& 4 	& 4 	& 5 & 1 &	4		& 3 			\\
		 	$\frac{p_i}{w_i}$ 	&	2	& 1.75 	& 1.5 	& 1.6 & 1 &	0.75	& $\frac{2}{3}$ \\
		 \end{tabular}
		 \\
		 Sei $C=15$ dann folgt:
		 
		 \begin{figure}[H]
		 	\centering
		 	\begin{tabular}{c | c c c c c}
		 		Iteration & nächster Gegenstand & passt? & add. Wert & $\sum$Wert & $C-\sum w$ \\ \hline
		 		0 & - & - & - & 0 & 15 \\
		 		1 & 2 & Ja & 4 & 4 & 13 \\
		 		2 & 3 & Ja & 7 & 11 & 9 \\
		 		3 & 4 & Ja & 6 & 17 & 5 \\
		 		4 & 7 & Ja & 8 & 25 & 0 \\
		 		5 & 6 & Nein & 0 & 25 & 0 \\
		 		6 & 1 & Nein & 0 & 25 & 0 \\
		 		7 & 5 & Nein & 0 & 25 & 0 \\
		 	\end{tabular}
		\end{figure}
	\end{figure}
	$\Rightarrow$ Die Lösung ist die Menge der Gegenstände $M=\{2,3,4,7\}$ mit einem Wert von $v=25$ und einer Restkapazität von $C_{Rest}=0$. Somit ist die Performance $R_H(P)=\frac{25}{25}=1$
	
	\subsection{b)} 
	Ja die Lösung ist optimal. Durch die vorherige Sortierung nach dem relativen Wert im Verhältnis zum Gewicht und da kein Spezialfall vorliegt lässt sich die Optimalität leicht erkennen.
	
	\subsection{c)}
	\begin{figure}[H]
		\centering
		\begin{tabular}{c | c c c c c}
			$i$ & 1 & 2 & 3 & 4 & 5 \\ \hline
			$p_i$ & 6 & 60 & 9 & 7 & 8 \\
			$w_i$ & 1 & 20 & 2 & 2 & 2 \\
			$\frac{p_i}{w_i}$ & 6 & 3 & 4.5 & 3.5 & 4 \\
		\end{tabular}
		$\overrightarrow{$sort wrt $\frac{p_i}{w_i}}$
		\begin{tabular}{c | c c c c c}
			$i$ 				& 1 & 3   & 5 & 4  & 2  \\ \hline
			$p_i$ 				& 6 & 9   & 8 & 7  & 60 \\
			$w_i$ 				& 1 & 2   & 2 & 2  & 20 \\
			$\frac{p_i}{w_i}$ 	& 6 & 4.5 & 4 & 3.5& 3  \\
		\end{tabular}
	
		\begin{figure}[H]
			\centering
			\begin{tabular}{c | c c c c c}
				Iteration & nächster Gegenstand & passt? & add. Wert & $\sum$Wert & $C-\sum w$ \\ \hline
				0 & - & - & - & 0 & 20 \\
				1 & 1 & Ja & 6 & 6 & 19 \\
				2 & 3 & Ja & 9 & 17 & 17 \\
				3 & 5 & Ja & 8 & 25 & 15 \\
				4 & 4 & Ja & 7 & 32 & 13 \\
				5 & 2 & Nein & - & 32 & 13 \\
			\end{tabular}			
		\end{figure}
	\end{figure}
	$\Rightarrow$ Die Lösung ist die Menge der Gegenstände $M=\{1 ,3 ,4, 5\}$ mit einem Wert von $v=32$ und einer Restkapazität von $C_{Rest}=13$. Somit ist die Performance $R_H(P)=\frac{32}{83}=0.3855$.\\ $z_{opt}=83$ erhalten wir mit der Menge $M_{opt}=\{1, 2, 3, 5\}$
	\subsection{d)}
	Das Ergebnis des Algorithmus bleibt das selbe, allerdings verändert sich der Optimale Wert auf $z_{opt}=84$ durch $M_{opt}=\{2,3,4,5\}$. Somit wird das Performance-Verhältnis noch schlechter.
	
	\subsection{e)}
	Für Greedy-Algorithmen können Szenarien konstruiert werden, in welchen sie relativ schlecht abschneiden. Solche Sonderfälle müssen dann zusätzlich abgefangen werden um die Performance zu verbessern. Insgesamt sind Greedy-Algorithmen im allgemeinen nicht optimal.
	
	\subsection{f)}
	Durch den Extended-Greedy-Algorithmus, in welchem am Ende nochmal geschaut wird für jeden \underline{nicht} mitgenommenen Gegenstand ob dieser, falls er alleine in den Rucksack passt, mehr Profit bringt wird der gesamte Inhalt durch eben jenen Gegenstand ausgetauscht.\\
	Somit würde der Extended-Greedy-Algorithmus an dieser Stelle für sowohl $C=25$ als auch für $C=26$ den Gegenstand 2 statt aller anderen in den Rucksack packen um auf einen Funktionswert von 60 zu kommen, welcher dann ein Performance-Verhältnis von $R_H(P_1)=\frac{60}{83}=0.7229$, bzw $R_H(P_2)=\frac{60}{84}=0.7143$ besitzt.
	
	\section{Aufgabe}
	\subsection{a)}
	
	\begin{figure}[H]
		\centering
		\begin{tabular}{c | c c c c c c c c}
			$i$ & 1 & 2 & 3 & 4 & 5 & 6 & 7 & 8 \\ \hline
			$w_i$ & 7 & 4 & 3 & 6 & 1 & 5 & 4 & 2\\
		\end{tabular} \\
	
		\begin{tabular}{c || c  c | c  c | c c | c c | c c}
			$bin/it.$ &  \multicolumn{2}{c}{1} & \multicolumn{2}{c}{2} & \multicolumn{2}{c}{3} & \multicolumn{2}{c}{4} & \multicolumn{2}{c}{5} \\ 
			 & $M_1$ & Rest & $M_2$ & Rest & $M_3$ & Rest & $M_4$ & Rest & $M_5$ & Rest \\ \hline\hline
			0 & $\emptyset$ & 8 &  $\emptyset$ & 8 &  $\emptyset$ & 8 &  $\emptyset$ & 8 &  $\emptyset$ & 8  \\
			1 & $\{1\}$ & 1	 	& $ $ & $ $ 		& $ $ & $ $ 		& $ $ & $ $ 	& $ $ & $ $ 	 \\
			2 & $ $ 	& $ $ 	& $\{2\}$ & 4 		& $ $ & $ $ 		& $ $ & $ $ 	& $ $ & $ $ 	 \\
			3 & $ $ 	& $ $ 	& $\{2,3\} $ & 1	& $ $ & $ $ 		& $ $ & $ $ 	& $ $ & $ $ 	 \\
			4 & $ $ 	& $ $ 	& $ $ & $ $ 		& $\{4\} $ & 2 		& $ $ & $ $ 	& $ $ & $ $ 	 \\
			5 & $\{1,5\}$ 	& 0	& $ $ & $ $ 		& $ $ & $ $ 		& $ $ & $ $ 	& $ $ & $ $ 	 \\
			6 & $ $ 	& $ $ 	& $ $ & $ $ 		& $ $ & $ $ 		& $\{6\} $ & 3 	& $ $ & $ $ 	 \\
			7 & $ $ 	& $ $ 	& $ $ & $ $ 		& $\{4,8\} $ & 0 		& $ $ & $ $ 	& $\{7\}$ & 4 	 \\
			8 & $ $ 	& $ $ 	& $ $ & $ $ 		& $ $ & $ $ 		& $ $ & $ $ 	& $ $ & $ $ 	 \\ \hline\hline
			$\sum$ & $\{1,5\}$ & 0 & $\{2,3\}$ & 1 & $\{4,8\}$ & 0 & $\{6\}$& 3 & $\{7\}$ & 4 \\
		\end{tabular}
	\end{figure}
	
	\subsection{b)}
	Nachdem sortieren sieht die Tabelle wie folgt aus:
	\begin{figure}[H]
		\centering
		\begin{tabular}{c | c c c c c c c c}
			$i$   & 1 & 4 & 6 & 2 & 7 & 3 & 8 & 5 \\ \hline
			$w_i$ & 7 & 6 & 5 & 4 & 4 & 3 & 2 & 1 \\
		\end{tabular}
	
	
		\begin{tabular}{c || c  c | c  c | c c | c c }
			$bin/it.$ &  \multicolumn{2}{c}{1} & \multicolumn{2}{c}{2} & \multicolumn{2}{c}{3} & \multicolumn{2}{c}{4} \\ 
			& $M_1$ & Rest & $M_2$ & Rest & $M_3$ & Rest & $M_4$ & Rest  \\ \hline\hline
			0 & $\emptyset$ & 8 &  $\emptyset$ & 8 &  $\emptyset$ & 8 &  $\emptyset$ & 8   \\
			1 & $\{1\}$ & 1	& $ $ & $ $ 		& $ $ & $ $ 		& $ $ & $ $\\
			2 & $ $ & $ $ 	& $\{4\}$ & 2 		& $ $ & $ $ 		& $ $ & $ $\\
			3 & $ $ & $ $ 	& $ $ & $ $			& $\{6\} $ & 3 		& $ $ & $ $\\
			4 & $ $ & $ $ 	& $ $ & $ $ 		& $ $ & $ $  		& $\{2\} $ & 4\\
			5 & $ $ & $ $	& $ $ & $ $ 		& $ $ & $ $ 		& $\{2,7\} $ & 0\\
			6 & $ $ & $ $ 	& $ $ & $ $ 		& $\{6,3\}$ & 0 		& $ $ & $ $\\
			7 & $ $ & $ $ 	& $\{4,8\} $ & 0 		& $ $ & $ $  		& $ $ & $ $\\
			8 & $\{1,5\} $ & 0 	& $ $ & $ $ 		& $ $ & $ $ 		& $ $ & $ $\\ \hline\hline
			$\sum$ & $\{1,5\}$ & 0 & $\{4,8\}$ & 0 & $\{6,3\}$ & 0 & $\{2,7\}$& 0 \\
		\end{tabular}
	\end{figure}
	Somit ergibt sich die min. Anzahl der Bins für die \textit{Best-Fit-Decreasing}-Heuristik bei 4.
\end{document}



















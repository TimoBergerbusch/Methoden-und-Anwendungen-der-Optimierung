\documentclass[12pt]{article}
\usepackage{amsfonts}
\usepackage{fancyhdr}
\usepackage[a4paper, top=2.5cm, bottom=2.5cm, left=2.2cm, right=2.2cm]{geometry}
\usepackage{times}
\usepackage{amsmath}
\usepackage{changepage}
\usepackage{amssymb}
\usepackage{graphicx}%
\setcounter{MaxMatrixCols}{30}
\newtheorem{theorem}{Theorem}
\newtheorem{acknowledgement}[theorem]{Acknowledgement}
\newtheorem{algorithm}[theorem]{Algorithm}
\newtheorem{axiom}{Axiom}
\newtheorem{case}[theorem]{Case}
\newtheorem{claim}[theorem]{Claim}
\newtheorem{conclusion}[theorem]{Conclusion}
\newtheorem{condition}[theorem]{Condition}
\newtheorem{conjecture}[theorem]{Conjecture}
\newtheorem{corollary}[theorem]{Corollary}
\newtheorem{criterion}[theorem]{Criterion}
\newtheorem{definition}[theorem]{Definition}
\newtheorem{example}[theorem]{Example}
\newtheorem{exercise}[theorem]{Exercise}
\newtheorem{lemma}[theorem]{Lemma}
\newtheorem{notation}[theorem]{Notation}
\newtheorem{problem}[theorem]{Problem}
\newtheorem{proposition}[theorem]{Proposition}
\newtheorem{remark}[theorem]{Remark}
\newtheorem{solution}[theorem]{Solution}
\newtheorem{summary}[theorem]{Summary}
\usepackage{enumitem}
\usepackage[utf8]{inputenc}
\newenvironment{proof}[1][Proof]{\textbf{#1.} }{\ \rule{0.5em}{0.5em}}
\usepackage{tikz,calc}
\usetikzlibrary{positioning,chains,fit,shapes,calc,arrows,patterns,external,shapes.callouts,graphs}
\usepackage{graphicx}
\usepackage{wrapfig}
\usepackage{float}
\usepackage{datetime}
\usepackage{ifthen}
\newdateformat{specialdate}{\twodigit{\THEDAY}.\twodigit{\THEMONTH}.\THEYEAR}
\usepackage[ngerman]{babel}
\usepackage{rotating}
\usepackage{hyperref}
\newcommand{\positiv}{~~\llap{\color{green!50!black}\textbf{+}\color{black}}~~}
\newcommand{\negativ}{~~\llap{\color{red}\textbf{-}\color{black}}~~}


\newcommand{\Q}{\mathbb{Q}}
\newcommand{\R}{\mathbb{R}}
\newcommand{\C}{\mathbb{C}}
\newcommand{\Z}{\mathbb{Z}}

\usepackage{pifont}% http://ctan.org/pkg/pifont
\newcommand{\cmark}{\ding{51}}%
\newcommand{\xmark}{\ding{55}}%

\begin{document}
	
	\title{8. Übung}
	\author{Timo Bergerbusch 344408}
	\date{\specialdate\today}
	\maketitle
	
	\section*{Aufgabe 1}
	\subsection*{a)}
		Aufteilung von $n=4$ Personen auf 4 Stühle ohne ""Zurücklegen" und mit BEachtung der Reihenfolge: $n!=4!=4\cdot 3\cdot 2 \cdot 1 = 24$
	\subsection*{b)}
%		Für n=4? oder Allgemein?
		Siehe \hyperref[Abb1]{Abbildung 1}. Sei die Tiefe fix mit $d_{\mathcal{N}_2}$
		\begin{figure}
			\centering
			
			\begin{tikzpicture}
				\draw (1*360/24: 8cm) node[circle, draw] (1234) {$1,2,3,4$};
				\draw (2*360/24: 8cm) node[circle, draw] (1243) {$1,2,4,3$};
				\draw (3*360/24: 8cm) node[circle, draw] (1324) {$1,3,2,4$};
				\draw (4*360/24: 8cm) node[circle, draw] (1342) {$1,3,4,2$};				
				\draw (5*360/24: 8cm) node[circle, draw] (1423) {$1,4,2,3$};
				\draw (6*360/24: 8cm) node[circle, draw] (1432) {$1,4,3,2$};
				
				\draw (7*360/24: 8cm) node[circle, draw] (2134) {$2,1,3,4$};
				\draw (8*360/24: 8cm) node[circle, draw] (2143) {$2,1,4,3$};				
				\draw (9*360/24: 8cm) node[circle, draw] (2314) {$2,3,1,4$};
				\draw (10*360/24: 8cm) node[circle, draw] (2341) {$2,3,4,1$};
				\draw (11*360/24: 8cm) node[circle, draw] (2413) {$2,4,1,3$};
				\draw (12*360/24: 8cm) node[circle, draw] (2431) {$2,4,3,1$};
				
				\draw (13*360/24: 8cm) node[circle, draw] (3124) {$3,1,2,4$};
				\draw (14*360/24: 8cm) node[circle, draw] (3142) {$3,1,4,2$};
				\draw (15*360/24: 8cm) node[circle, draw] (3214) {$3,2,1,4$};
				\draw (16*360/24: 8cm) node[circle, draw] (3241) {$3,2,4,1$};				
				\draw (17*360/24: 8cm) node[circle, draw] (3412) {$3,4,1,2$};
				\draw (18*360/24: 8cm) node[circle, draw] (3421) {$3,4,2,1$};
				
				\draw (19*360/24: 8cm) node[circle, draw] (4123) {$4,1,2,3$};
				\draw (20*360/24: 8cm) node[circle, draw] (4132) {$4,1,3,2$};				
				\draw (21*360/24: 8cm) node[circle, draw] (4213) {$4,2,1,3$};
				\draw (22*360/24: 8cm) node[circle, draw] (4231) {$4,2,3,1$};
				\draw (23*360/24: 8cm) node[circle, draw] (4312) {$4,3,1,2$};
				\draw (24*360/24: 8cm) node[circle, draw] (4321) {$4,3,2,1$};
				
				%TODO edges				
				
			\end{tikzpicture}
			\label{Abb1}
			\caption{test}
		\end{figure}
		
	\subsection*{c)}
		Maximal braucht man $d_{\mathcal{N}_2}$ Iterationen in der lokalen Suche um zur besten, mittel $\mathcal{N}_2$ erreichbaren, Lösung zu kommen.
	\subsection*{d)}
		Da ein Tisch $n=4$ Personen hat und diese in 24 gültigen Reihenfolgen sitzen können müssen 24 Swap-Chair-Moves pro Tisch in jedem Schritt der lokalen Suche betrachtet werden. Somit insgesamt 48.
	\subsection*{e)}
		
		\begin{tabular}{r | c c c c}
			& $T_1$ & $f(T_1)$ & $T_2$ & $f(T_2)$ \\ \hline
			Vorm Schütteln & $\{a,d,c,e\}$ & & $ \{b,f,g,h\}$ & \\
			Nach Schütteln($\mathcal{N}_1$) & $\{g,d,c,e\}$ & & $ \{b,f,a,h\}$ & \\
			Nach lok. S. & $\{g,f,c,e\}$ & & $ \{b,d,a,h\}$ & \\
		\end{tabular}
		
		
		
\end{document}


















